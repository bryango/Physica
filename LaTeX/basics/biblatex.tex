% biblatex settings
% ... extracted from `pkuthss`
% ... biber usage:
%%%% !TeX TXS-program:bibliography = biber -l zh__pinyin --output-safechars %
%%%% % 注意末尾的 %, 适用于 TeXstudio

%% ... include bib:
%\addbibresource{*.bib}
%% ... print bib:
%\printbibliography[heading = bibintoc]
%% bibintoc 选项使“参考文献”出现在目录中
%% 如果同时要使参考文献列表参与章节编号,可将“bibintoc”改为“bibnumbered”

\usepackage[utf8
%	,style=caspervector
%	,backend=biber % caspervector 必须使用 biber 后端
%	,sorting = ecnty % 英、中文献排序,对比 none, centy
]{biblatex}
% 按学校要求设定参考文献列表中的条目之内及之间的距离
\setlength{\bibitemsep}{3bp}
% linespread 值的计算过程可以参考 pkuthss.cls
\renewcommand*{\bibfont}{\zihao{5}\linespread{1.27}\selectfont}
